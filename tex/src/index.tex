% roman page numbers on preface
\pagenumbering{roman}

\vspace*{2.5cm}
\section*{Preface to this Edition}
\vspace*{4cm}

This text is an English translation of the key chapter of a thesis I wrote in
German as a philosophy student in the early 1990s. At the time, I tried to align
anarchist beliefs with postmodern/poststructuralist theory and cultural studies.
Not long thereafter, I lost interest in these pursuits. Today, I’m baffled by
the nonsense that has been produced (and continuous to be produced) under the
“post” banner. However, I still believe that the theorists who inspired these
fields (often against their will) made valuable contributions to the history
of ideas and intervened at a time when social sciences and humanities needed a
breath of fresh air.

The thesis in question was called \textit{Jenseits von Staat und Individuum.
Studien zu alteritärer Individualität} (Beyond the State and the Individual:
Studies on Alternative Individuality). In 2007, a revised version with a less
pretentious subtitle (Individuality and Autonomous Politics) was published by
Unrast Verlag. I have, at times, called this my most underrated book. That’s
tongue-in-cheek, but even if there has been an appreciative audience, it has
always remained limited. I believe, though, that the book targets a problem that
has only grown since then, namely: how are politics based on collective values
possible in an individualized world? Especially, when we, the “subjective
forces” of the revolution, have a hard time escaping our own individualization
as well. The issue is very personal, as my political beliefs revolve around
mutual aid and communal efforts, while I find alone time and personal
independence very precious. Is this a contradiction that can be resolved? Are
there ways to reconcile these apparent opposites? Or are our collective
aspirations doomed?

\hyphenation{Le-ga-cy}
In \textit{Jenseits von Staat und Individuum}, these questions were mainly
discussed with references to popular culture, everything from literary classics
to Italo Westerns to straight edge. Yet, the chapter “Das Vermächtnis des
Anarchismus: Skizze einer antiindividualistischen Individualität” (The Legacy of
Anarchism: Sketching an Antiindividualistic Individuality) tried to tackle them
theoretically.

In 2003, I translated this chapter into English and published it as a pamphlet
under the title “Antiindivdualistic Individuality: A Concept” in the Alpine
Anarchist series. Alpine Anarchist Productions was a DIY publishing project I
ran at the time. As with most Alpine Anarchist releases, I used a pseudonym for
this pamphlet: Teoman Gee, the most common one.

In 2007, I stopped releasing pamphlets and moved Alpine Anarchist Productions
online. While I uploaded a fair number of the pamphlets I had released, I never
uploaded “Antiindividualistic Individuality”, because I was unhappy with parts
of the translation.

It was therefore a very pleasant surprise when, in July 2023, the editors of
Laniakea Books got in touch and expressed interest in rereleasing the pamphlet.
This goes to show that even pamphlets released in very small numbers can have
long-lasting effects. (Shout out to all the zine makers out there who still
stick to the trade!) So, I gave the translation an overhaul and handed it to
the Laniakea folks to produce the booklet you’re holding in your hands (or
peruse online). There are features that you wouldn’t find in current texts of
mine (quite a bit of the wording and neverending notes!), but I didn’t change
any of that. The document stands for its time. Many thanks for the opportunity
to see it in print again!

\bigskip

\noindent\emph{Gabriel Kuhn, Stockholm, August 2023}

\clearpage

% main text (after preface)
\pagenumbering{arabic}
\setcounter{page}{1}

\newpage

\vspace*{2.5cm}
\section*{Antiindividualistic Individuality:\\Anarchism’s Legacy}
\vspace*{4cm}

\begin{adjustwidth}{5em}{0pt}
    \setlength{\parindent}{0pt}
    \setlength{\parskip}{0.5\baselineskip}

    “The new man will find himself only when the warfare between the collective
    and the individual ceases. Then we shall see the human type in its fullness
    and splendor.”

    — Henry Miller, \emph{The Time of the Assassins}
\end{adjustwidth}
\vspace{2\baselineskip}

There can be no critique of the prevailing bourgeois-capitalist ideology without
a critique of modern individualism. For it is modern individualism that leads us
to perceive ourselves as the sole masters of our destiny, thrown into this world
only to see to our own personal benefit. Under these ideological conditions we
find ourselves united with others only in the abstract entity of the state. This
allows the state to take control over us, that is, our lives and our spirits.
Our “freedom” is reduced to the \textit{war of all against all} over social,
political, and economic resources, in other words: over power. We end up as
subjects to the state, and as enemies to our brothers and sisters, while we
flatter ourselves to be “autonomous” from everything that surrounds us — a
destructive concept in itself. This is the situation that modern individualism
has put as in. In order to overcome the state and capital, this situation has to
be overcome as well.

This does not mean, however, that we can do without a theory of individuality.
For even if the individual\textsuperscript{1} reinvents its identity as a
collective being, it doesn’t explain what the individual’s \textit{role} in the
collective will be.\textsuperscript{2} Put differently, in the aspired
collective forms of living, questions about individuality won’t disappear. To
the contrary: reflecting upon the social significance of individuality will be
of great importance both for the collective to function and for the individuals
to make valuable contributions. In short, individuality is a vital aspect of
making antiindividualistic communities possible.

As much as the collective will define the roles of the individuals, the
individuals will define the collective. African scholar Fodé Diawara wrote in
1972: “The strong sense of community, characteristic for the ‘primitive
mentality’ (something no anthropologist can seriously deny) is not based on
instinct. It is first and foremost the result of ongoing reflection within both
the individual and ‘primitive’ social structures.”\textsuperscript{3}

We need to distinguish between the following: when individuals live in a
collective web, they \textit{blend} into it, but they don’t \textit{dissolve} in
it. It is a dangerous misconception to believe that in an antiindividualistic
society individuals will cease to exist. Rather, the individual will have
special value for the community, which thrives on people’s individuality. This
is why an antiindividualistic community does \textit{not} need \textit{no}
concept of individuality, but an \textit{antiindividualistic}
one.\textsuperscript{4}

Unfortunately, this distinction is not always made in critical theory. People
tend to cling to the bourgeois separation of the individual and society. In this
framework, any antiindividualistic effort will consist of taking society’s side.
This, however, only means to flip the same coin upside down. When alternative
forms of individuality replace individualism, alternative forms of collectivity
will have to replace the state. While the individual and society reinforce each
other as the ideological basis of the bourgeois-capitalist state, we are looking
for individuality and collectivity \textit{beyond the state and the modern
individual}.

The social significance of individuality unfolds in communities that do not know
of the individual vs. society dichotomy. There, individuals to come to life in
the collective, just as the collective comes to life in the individuals. In
other words, individuality can only unfold its social value where there is
constant interaction on one and the same social terrain.

Leftist theory often disregards the necessity of alternative concepts of
individuality. Sometimes because the problem isn’t taken seriously, and
sometimes because of the assumption that it will solve itself in a socialist
society. This is one of the reasons why leftist notions of collectivity can get
stuck in abstract utopianism without much practical relevance (in the best
case), or lead to totalitarian regimes (in the worst).

It’d be unfair to accuse Karl Marx of “forgetting the individual.” Marx hoped
for a liberation of the individual in communism. In the
\textit{Economic-Philosophical Manuscripts} he says: “We have to avoid
juxtaposing the ‘society’ as an abstraction to the
individual.”\textsuperscript{5} But his trust in
historic necessity might have got in the way when it came to developing concrete
concepts of individuality. Without such concepts, however, the individual will
remain enslaved even when the means of production are in the hands of the
masses.

Marxism does not really offer a theory of alternative
individuality.\textsuperscript{6} The following comment of a former German Red
Army Faction guerilla seems revealing: “In the RAF we thought that we’d
automatically change ourselves in the struggle for liberation, in the struggle
for changing society’s objective conditions. This is certainly false. The
personal struggle has to be part of any liberation struggle.”\textsuperscript{7}

In this context, it seems impossible not to pay tribute to the anarchists. The
theory of classical anarchism might not be all that deep, but it drew the line
to authoritarian socialism where it sensed the danger of abstract models of
collectivization — which, despite Marx’s warnings, did develop in mainstream
Marxist-Leninist ideology.

The anarchists took an uncompromising and uncorruptible stand against this.
Bakunin reminded us that “collective liberty and prosperity exist only so far as
they represent the sum of individual liberties and
prosperities.”\textsuperscript{8} Erich Mühsam said that “if the individualists,
who see nothing but the independence of the individual, can’t see the forest for
the trees, certain collectivists aren’t any better: all they see is the forest,
and they refuse to realize that it consists of trees.”\textsuperscript{9} Diego
Abado de Santillan optimistically predicted on the eve of the Spanish
Revolution: “The eternal aspiration to be unique will be expressed in a thousand
ways: the individual will not be suffocated by levering down. … Individualism,
personal taste, and originality will have adequate scope to express
themselves.”\textsuperscript{10} Errico Malatesta summarized the importance of
individuality thus: “It is uncontested by anarchists that the real, concrete
being, the being who has consciousness and feels, enjoys, and suffers, is the
individual, and that society (far from being superior to the individual, far
from seeing the individual only as an instrument and a slave) must be no more
than the union of associated men and women for the greater good of
all.”\textsuperscript{11} Malatesta believed that anarchists are “the militant
custodians of liberty against all aspirants to power and against the possible
tyranny of the majority.”\textsuperscript{12}

The following attempt to understand individuality as a \textit{social value}
(or, to outline this social value theoretically) is motivated by the conviction
that a concept of individuality is crucial in our struggle for
antiindividualistic forms of living. What follows is therefore the attempt to
conceptualize an \textit{antiindividualistic individuality} aiming “to
rediscover a relation to oneself as a free individuality.”\textsuperscript{13}

It won’t be enough, of course, to stress concepts from Aristotle’s \textit{zoon
politikon} to Heidegger’s \textit{In-der-Welt-Sein} as Mitsein
(Being-in-the-World as Being-With), defining the human being as a per se “social
being.” No one really questions this, and we are not pursuing abstract
philosophical anthropology here,\textsuperscript{14} but rather concrete,
practical ways of existence and related ethical principles. Whenever we’re faced
with grand philosophical theory, we should recall the words of Bakunin: “Not
that it is ignorant of the principle of individuality; it conceives it perfectly
as a principle, but not as a fact.”\textsuperscript{15} We need to think of
individuality as the \textit{radical materialistic manifestation of our
uniqueness}.\textsuperscript{16}

Individuality as a social value has nothing to do with the modern individual.
The modern individual is a rigid entity, individuality a dynamic principle.
These are opposites. While the modern individual stands for conformism and
universalism, individuality stands for anticonformism and antiuniversalism. To
quote Ralph Waldo Emerson: “Whoso would be a man, must be a
nonconformist.”\textsuperscript{17}

Individuality is rated highly in antistatist collectives. It guarantees both the
immanence and the permanence of social uniqueness. Timothy Leary illustrated
this by saying: “Self-assured singularities … have been called mavericks, ronin,
freelancers, independents, self-starters, nonconformists, oddballs,
troublemakers, kooks, visionaries, iconoclasts, insurgents, blue-sky thinkers,
loners, smart alecks … they have been variously labeled clever, creative,
entrepreneurial, imaginative, enterprising, fertile, ingenious, inventive,
resourceful, talented, eccentric.”\textsuperscript{18}

If we want to make use of the terminology offered by Gilles Deleuze and Félix
Guattari, we could say that individuality is a precondition for
\textit{molecular groups}, which are characterized by diversity, vitality, and
motion, while it is suppressed in the rigid, drab, and one-dimensional
\textit{molar groups} institutionalized by the state.\textsuperscript{19}
Carlos Marighela wrote in the \textit{Minimanual of the Urban Guerrilla}: “The
characteristics of the urban guerilla are initiative, creativity, flexibility,
diversity, and quick-wittedness.”\textsuperscript{20}

Individuality’s function in antistatist collectives can be summarized by seven
points, outlining a concept of antiindividualistic individuality.

1. To understand oneself as oneself, yet at the same time as an integral part of
a collective, demands \textit{individual responsibility} for the collective. If
people live together not governed by the state but relying on principles that
have been negotiated collectively, they can’t take their “personal rights” and
lock themselves into a chamber. Instead, they will have to contribute actively
to the survival of the group.

Foucault talked about the \textit{polis} in ancient Greece as a complex,
pre-statist form of governance (even if reserved for a privileged minority):
“The attitude of the individual to itself, the way in which it guards its
freedom in relation to its desires, the sovereignty which it enacts upon itself,
are constitutive elements of the happiness and order of the
\textit{polis}.”\textsuperscript{21}

A rigid social order requires passive individuals, while a vibrant social order
requires active individuals. The social tasks that the bourgeois-capitalist
order transfers to the state (education, health care, control of public space,
conflict solution, etc.) must, in an antistatist collective, be met by a
coordinated effort of individuals. The same is true for our economic tasks. This
requires individuals willing to accept their social responsibility, ready to
take on relevant tasks.

In \textit{The Philosophy of Punk}, Craig O’Hara summarized anarchy thus:
“Anarchy does not simply mean no laws, it means no need for
laws.”\textsuperscript{22} The fact that the state legitimizes itself by
claiming that it is needed to administer collective life (allegedly, because the
individuals are not capable of it) illustrates the revolutionary character of
this principle. Fellow anarchist Alex Comfort has stated that “anarchism is that
political philosophy which advocates the maximization of individual
responsibility.”\textsuperscript{23} Crime novelist Léo Malet has identified
anarchists thus: “Anarchists are those who try to think their own thoughts. They
are special people, even the dumbest among them. They have something the
ordinary citizen, the trained voter, doesn’t have: they are more original than
a lamppost.”\textsuperscript{24}

Individual responsibility implies to always reflect upon one’s actions and to
withstand fascist obedience. In the words of Ernst Bloch, “extinguished
individuals deprived of responsibility” don’t just create “imbeciles but
beasts.”\textsuperscript{25} Or, as an autonomous activist from Germany has put
it: “If self-determination means to take on responsibility for consequences, it
will create a different consciousness, making it impossible for people to act as
pigs. Both the positive and negative effects of your actions will get back to
you.”\textsuperscript{26}

2. Related to the notion of individual responsibility is the notion of
\textit{self-respect}. Only those able to see themselves as agents in the social
arena (and not as remote-controlled clones of a frightening social machine) can
develop a secure and positive self-image.

Ethics play a role, of course. Bourgeois agents are agents, too, and they might
develop a secure and positive self-image, too, but this will be based on the
narcissistic self-satisfaction of people pursuing money and fame. It can make
some people “proud,” but no true \textit{self-respect} can come from this.
Self-respect is necessary, however, if we don’t want to hide
paranoically\textsuperscript{27} behind abstract identities — identities that
allow us to compensate for feelings of inferiority by attacking those we
perceive to be the weaker.

Self-respect has nothing to do with arrogance. That’s for the bourgeoisie. When
we speak of self-respect, we mean levelheadedness, calm, and
sovereignty;\textsuperscript{28} essential values for people who desire to live
in antistatist communities, in which antiindividualistic individuality
flourishes.

What follows are three references to the importance of self-respect. The rapper
Ice-T once told the following story: “I asked Everlast, who raps with House of
Pain, if he thought white people hate black people. … He doesn’t think white
people hate black people, he just figures they’re afraid of them. … And the
longer we live in the attitude that it’s cool to oppress people, the longer
we’re going to be afraid.”\textsuperscript{29} Henry Rollins has said: “Anyone
with real self-respect wouldn’t litter, wouldn’t steal, wouldn’t lie, wouldn’t
fuck anyone else over. He wouldn’t do anything to anyone else he wouldn’t do to
himself. When I respect myself, I don’t need to take out my pain or weakness on
someone else.”\textsuperscript{30} And the Beastie Boys have rapped on
\textit{Ill Communication}: “We’re here to work it out in one way or another /
to find a mutual respect for ourselves and one another / and the true key is a
trust in self / for when I trust myself I fear no one else.”\textsuperscript{31}

3. From the \textit{Discover Yourself!} inscribed in the temple walls at Delphi
to Henry Rollins’s comment that “the more you know about yourself, certainly the
less harm you’re going to do to other people,”\textsuperscript{32} there exists
an important line of thought. It is not about mystifying a “true self” that
needs to be rediscovered, but about a tradition of \textit{pragmatic}
anthropology, since, as Gernot Böhme has put it, “the path of anthropology in a
pragmatic sense reveals the kind of human being you are.”\textsuperscript{33}
In simpler terms, it’s about being honest to yourself. This honesty about
\textit{how} (and not necessarily \textit{what}\textsuperscript{34}) we are, is
important in many ways: for self-reflection (without which we won’t come to a
new understanding of individuality or the ability to change ourselves), for
authenticity in social relationships, and for constructiveness in social
dialogue.

4. Individuality guarantees \textit{creativity} and \textit{ingenuity} in daily
life, in how we make decisions, solve problems, and organize. John Stuart Mill
observed that life comes to a standstill “when it loses
individuality.”\textsuperscript{35} There is no vital community without
creativity.

5. Individuality \textit{prohibits totalitarianism}. It is the only guarantee
for critical intervention. It is also the most effective social protection
against the concentration of power and the institutionalization of authority.
Max Stirner is a controversial thinker, but his reminder that the individual is
“the irreconcilable enemy of any universality, meaning: any yoke” is
important.\textsuperscript{36}

6. Individuality guarantees the diversity of talent and skill necessary to
fulfill the plethora of social tasks. In any collective, various social tasks
need to be met, and various roles need to be taken. If \textit{immanent
diversity} ought not give way to statist conformity, individuality must be
defended, since, without it, a radiant and diverse social life is not possible.

7. Finally, individuality is a requirement for \textit{social dynamism}. I’ll
try to explain, using an important terminological distinction in
poststructuralist theory, namely, the one between \textit{subject} and
\textit{subjectivity}. The subject is about rigid identity; subjectivity is
about dynamic creation — or, as Gilles Deleuze has put it, “subjectivity is the
production of forms of existence.”\textsuperscript{37}

Subjectivities don’t produce subjects. The subject stands in the way of
subjectivity. The subject is an entity; subjectivity is a process. Subjectivity
threatens the subject, whose aim it is to make life a prisoner. But subjectivity
has nothing to do with arbitrary forms of identity. Arbitrary forms of identity
provide a consumerist façade for the modern individual, allowing it to justify
anything and turn hypocrisy into a virtue.\textsuperscript{38}

Social dynamism refers to something very different, namely, the individuals’
ability to account for and reflect on the dynamic character of reality. The
reality we live in, socially and ecologically, is constantly changing, and
therefore our position in it is changing, too. As responsible individuals, we
have to reflect on these changes, redefine ourselves, and adapt our communities.
Flexibility is a requirement for this. Active individuals will always stand in
opposition to passive subjects of the state, no matter how much their
passiveness is hidden behind a broad range of consumerist
choices.\textsuperscript{39}

Following antiindividualistic ideals, we, as individuals, must understand
ourselves as social singularities and not as a subject of the state. Social
singularity is defined by nonconformism and a genuine opposition to
universalism. We have looked at seven values that are essential for antistatist
communities and that are dependent on embracing individuality: individual
responsibility, self-respect, self-discovery, creativity and ingenuity,
antitotalitarianism, immanent diversity, and social dynamism. These values
cannot wait for some utopia to be realized, they need to be realized here and
now, as we are trying to build antiindividualistic and antistatist forms of
living. This requires distancing ourselves from indoctrinating powers and social
norms, from mass media and “public opinion,” from “schools” of all sorts, as
they pose a huge danger to individuality. If they manage to indoctrinate us, we
lose our uniqueness. We must resist becoming the conformist mass being that the
bourgeois-capitalist order wants us to be. Gilles Deleuze has put it thus: “The
fight for subjectivity is fought by resisting both prevailing forms of
submission: one consists of individualizing us according to the demands of
power, the other of tying each individual to a predetermined identity. The fight
for subjectivity thus presents itself as a right to difference, to variation, to
metamorphosis.”\textsuperscript{40}

Individual rebellion must be understood in this light. It occurs when
individuals dismantle conformist identities and become singularities. This must
not be mistaken for individual\textit{ism}. Individual rebellion interrupts the
individualistic-conformist order of the state. It’s no coincidence that the
lives of punks, squatters, and autonomists are frequently referred to as
“chaotic” in bourgeois media, when they simply are creative.

Individual rebellion demands a subversion of the modern individual much rather
than being its ultimate glorification.\textsuperscript{41} It is unfair to
accuse thinkers such as Julia Kristeva, who fought for singularity and against
monolithic social entities, of “individualism.”\textsuperscript{42} Such attacks
miss their target completely. All that Kristeva did was what all nonconformists
do: she demanded individuality as a social singularity in the name of
antiindividualistic and antistatist collectivity.\textsuperscript{43} Her
struggle belonged to the ones described by Michel Foucault: “They are struggles
which question the status of the individual: on the one hand, they assert the
right to be different, and they underline everything which makes individuals
truly individual. On the other hand, they attack everything which separates the
individual, breaks his links with others, splits up community life, forces the
individual back on himself, and ties him to his own identity in a constraining
way. These struggles are not exactly for or against the ‘individual’ but rather
they are struggles against the ‘government of
individualization.’”\textsuperscript{44}

Antiindividualistic praxis does not aim \textit{to liberate the individual from
the state}; it is directed \textit{against the production of the individual by
the state}. To quote Foucault once more: “The conclusion would be that the
political, ethical, social, philosophical problem of our days is not to try to
liberate the individual from the state and from the state’s institutions but to
liberate us both from the state and from the type of individualization which is
linked to the state.”\textsuperscript{45}

This recalls a differentiation made by Peter Kropotkin already a century ago:
“When the mutual aid institutions — the tribe, the village community, the
guilds, the medieval city — began, in the course of history, to lose their
primitive character, to be invaded by parasitic growths, and thus to become
hindrances to progress, the revolt of individuals against these institutions
took always two different aspects. Part of those who rose up strove to purify
the old institutions, or to work out a higher form of commonwealth, based upon
the same mutual aid principles. … But at the very same time, another portion of
the same individual rebels endeavored to break down the protective institutions
of mutual support, with no other intention but to increase their own wealth and
their own powers.”\textsuperscript{46}

\hyphenation{pro-test}
In contemporary terminology, this means that the individuals who demand more and
more rights for themselves, such as libertarians, have nothing to do with
individual protest for the sake of total liberation; they only try to maximize
individualistic gains.

Right-wing libertarians want to see the bourgeois individual reach its ultimate
power. It is supposed to become so powerful that it no longer needs the state to
protect it. Here, the rejection of the state stands for the rejection of all
individual responsibility for the collective. Essentially, it becomes a
glorification of social Darwinism, evident in large parts of the US militia
movement.

The intention of the antiindividualists is of a different kind: they want to
break the abstract collective monopoly of the state to make way for lived
collectivities. Their individual protest wants to escape the statist prison.
While the rightist libertarian rejection of the state is tied to
ultra-individualism, antisocialism, and Ramboism, the antiindividualistic
rejection of the state stands for social responsibility, collective
consciousness, and antifascism.

Of course, this difference is expressed in praxis, too. Even if slogans like
“No government is the best government!” might be heard from both sides, the
individualistic and the antiindividualistic approach, respectively, lead to very
different forms of praxis. While the individualists reject \textit{any} measures
taken by the state at \textit{any} time, because they see their “individual
freedom” threatened, antiindividualists know how to differentiate between
statist surveillance, control, and indoctrination on the one hand, and social
policies necessary under the given circumstances on the other. It is unfortunate
that, today, certain social necessities can only be provided by the state.
Antiindividualists won’t necessarily oppose welfare programs or subsidies for
education, health care, public transport, and housing, if the only alternative
lies in more private power, more egoistic competition, and less social
commitment.

This does not mean that antiindividualists don’t try to create alternatives to
the social monopolies of the state. To the contrary, this very activity defines
the second major difference to the antistatism from the right. While the latter,
in Wild West romanticism, relies on the supposedly just power of the gun to get
what you need, antiindividualist and antistatist resistance aims to create
communities based on social responsibility and everyday collaboration, allowing
a life independent of the state’s social policies, which demand individual
submission rather than responsibility.

There are numerous misunderstood individualists. We could think of people as
different as Albert Camus or Chuck Bukowski. They had to share the fate of the
dadaists or surrealists, who allegedly got themselves caught in contradictions
between individual actionism and collective political commitment. It is sad that
\textit{individual resistance in the spirit of collectivity} often gets
misunderstood as \textit{individualistic} resistance. It is also sad that its
revolutionary potential is often not recognized. And it is inexcusable that it
has often been attacked as counter-revolutionary by leftists. This has done much
harm to revolutionary movements. The ideals of the nonconformists are essential
for an antifascist life.\textsuperscript{47} The answer to an individualistic
society is not a totalitarian one (that’s just flipping the coin), but forms of
existence in which individuality and collectivity complement each other.
Individualism doesn’t end where conformism begins; it ends where individuality
and collectivity merge and create new forms of community.

Only fascism and totalitarianism need conformism. \textit{Antiindividualistic
individuality} is a requirement for vibrant and diverse collectivity, formed in
daily praxis. It is embedded in a spirit of solidarity and antifascism. If these
elements are missing, “individuality” can quickly deteriorate into classicist
cults of the genius, bourgeois dandyism, or careerist yuppiedom. There won’t be
individuality for all without social equality. This is what the anarchists
remind us of.

In this context, individual freedom has nothing to do with retreating to a
secluded cell. Michel Bakunin: “For a human being to be free means to be
recognized, regarded, and treated as such by a fellow human being, by all human
beings. Freedom does not mean isolation but mutual recognition; it does not mean
separation but unification.”\textsuperscript{48} Walter Benjamin (not an
anarchist) has claimed that “since Bakunin, Europe has lacked a radical concept
of freedom.”\textsuperscript{49}

As Félix Guattari has noted, “the individuals have to develop solidarity and
diversify at the same time.”\textsuperscript{50}

\newpage

\vspace*{2.5cm}
\section*{Notes}
\vspace*{4cm}

{
    \setlength{\parindent}{0pt}
    \setlength{\parskip}{\baselineskip}

    1. In German, there is a useful terminological distinction between
    \textit{Individuum} und \textit{Einzelner}. \textit{Einzelner} refers to a
    single being (you, me, Peter, Paul, and Mary); it is a modest term with no
    ontological implications. \textit{Individuum}, on the other hand, suggests
    that there is an essence to you, me, Peter, Paul, and Mary, that we are
    separated as ontological entities. There is no real equivalent for this
    distinction in English, as a commonly established word for
    \textit{Einzelner} is lacking. When, in this text, I speak of the
    “individual” in an “antiindividualistic” sense, I mean \textit{Einzelner}.

    2. There are two historical texts of interest in this context: 1.
    \textit{The Right to be Greedy}, written by a group named \textit{For
    Ourselves} in 1974, brings together Karl Marx, Max Stirner, Raoul Vaneigem,
    and others, in an attempt to develop an “individualist communism” that is
    also a “communist individualism,” based on the conviction that “developed
    individuals make a richer community, and a richer community makes for richer
    individuals.” 2. \textit{Max Stirner und der moderne Sozialismus}, a 1906
    essay by Austrian socialist Max Adler, attempting a similar combination of
    individualism and socialism: “Only those who do not understand individualism
    as creation of free personality, but as carelessness without personality
    (something no anarchist who wants to be taken seriously would do), will see
    it in opposition to socialism. In its true nature individualism is a basic
    principle of socialism, explicitly striving for the creation of a society in
    which the \textit{free development of each and every one is the precondition
    for the free development of all}.”

    3. Fodé Diawara, \textit{Le manifeste de l'homme primitif} (1973).

    4. Michel Foucault: “We need to create new forms of subjectivity by
    rejecting the form of individuality forced upon us for centuries”
    (\textit{The Subject and Power}, 1982). Ernesto Laclau and Chantal Mouffe:
    “What’s at stake her is the creation of a different individual, an
    individual that’s no longer constructed by the matrix of possessive
    individualism” (\textit{Hegemony and Socialist Strategy: Towards a Radical
    Democratic Politics}, 1985).

    5. Karl Marx, \textit{Economic and Philosophical Manuscripts}, 1844.

    6. If we take, for example, Gottfried Stiehler’s \textit{Über den Wert der
    Individualität im Sozialismus}, published in East Germany in 1978, we are
    told that only socialism “allows for the creation of free individuality.”
    The how and why is explained by “the objective essence of socialism itself.”
    Stiehler adds: “The free and all-encompassing development of the individual
    is no mere rhetorical phrase in socialism, but an objective necessity.”

    7. Margret, in: \textit{Der Stand der Bewegung} (1995).

    8. Mikhail Bakunin, \textit{God and the State} (1871).

    9. Erich Mühsam, \textit{Fanal} (ca. 1928).

    10. Quoted from Daniel Guérin, \textit{Anarchism: From Theory to Practice}
    (1970).

    11. Errico Malatesta, “Communism and Individualism” (1926).

    12. Errico Malatesta, “The Anarchist Revolution” (1924).

    13. Gilles Deleuze, \textit{Foucault} (1986).

    14. Those who are interested could look at Gernot Böhme’s
    \textit{Anthropologie in pragmatischer Hinsicht} (1985).

    15. Bakunin, \textit{God and the State}. Bakunin explains further: “Human
    individuality, just like the individuality of the most motionless things,
    can’t be comprehended by science and hence doesn’t exist for it. That’s why
    the living individualities have to protect themselves (against science) not
    to be sacrificed to some abstraction like a guinea-pig; the same way they
    have to protect themselves against theology, politics, and law, all of which
    correspond to the abstract character of science and have the dangerous urge
    to sacrifice the individuals to some abstraction, disguised only by
    different names: theology calls it ‘divine truth,’ politics ‘the public
    good,’ law ‘justice.’” (ibid.)

    16. This explains why there is no discussion here about the “essence” of the
    human being, about the “self,” “self-awareness,” or “personality”. These are
    very “heavy” terms. In this text, individuality refers to singularities
    (“individuations”), an integral part of life, which we experience as
    “facts.” Whether there is anything essential behind them – or what such an
    essence would consist of – is not the issue here. If someone demanded the
    necessity for a philosophical investigation into what “I” really means, we
    could answer with the German anthropologist Hans Peter Duerr, who wrote:
    “We simply learn to use the word ‘I’ through learning to partake in
    conversation, to address other people and to be addressed. And should
    someone ask skeptically, ‘How can you ever know what I mean when I say
    \textit{I}?,’ the response would be: ‘Just as I know what you mean when you
    say \textit{you}!’” (\textit{Ni dieu – ni metre}, 1985).

    17. Ralp Waldo Emerson, \textit{Self-Reliance} (1841).

    18. Timothy Leary, \textit{Chaos \& Cyberculture} (1994).

    19. See: Gilles Deleuze and Félix Guattari, \textit{Anti-Oedipus} (1977).

    20. Carlos Marighela, \textit{Minimanual of the Urban Guerrilla} (1969).

    21. Michel Foucault, \textit{The Use of Pleasure} (1984).

    22. Craig O’Hara, \textit{The Philosophy of Punk} (1992).

    23. Alex Comfort, preface in: Harold Barclay, \textit{People without
    Government} (1982). A quote by Errico Malatesta explains how anarchists
    manage to overcome the difference between “egoistic” and “altruistic”
    behavior: “This now century-old dispute between ‘egoists’ and ‘altruists’ is
    in fact nothing but an unfortunate fight over words. It’s evident and
    admitted by all that anything we do voluntarily, we do to please our senses,
    or to follow our convictions. Even the purest martyr sacrifices himself
    because he experiences a deep satisfaction in the act of sacrifice, which
    rewards him plentifully for the experienced suffering; and if he gives his
    life consciously and voluntarily, then because, in his eyes, there is
    something of higher value than life. In a certain sense we can therefore,
    with no fear to go wrong, say that all humans are egoists. In colloquial
    speech, however — which, in my opinion, always has to be preferred, as long
    as it doesn’t cause misunderstanding — an egoist is someone who only thinks
    of himself and sacrifices the interests of others to his own, while an
    altruist is someone who is also — more or less strongly — concerned about
    the interests of others and does anything he possibly can to help them. In
    short, the egoist would be the worse egoist, and the altruist the better
    egoist — a question of words.” (Errico Malatesta, “The Moral Foundation of
    Anarchism,” 1922)

    24. Léo Malet, \textit{Des kilomètres de linceuls} (1955).

    25. Ernst Bloch, \textit{The Principle of Hope} (1954).

    26. Dieter, in: \textit{Der Stand der Bewegung}. – That the conditions for
    this aren’t the best in an individualized society since contact between
    individuals is loose enough to shed any social responsibility, is being
    pointed out by Hans Peter Duerr: “Certainly the individual today is
    connected to more people than ever before, yet the partners of interaction
    aren’t so much ‘whole personalities’ than ‘fragments of persons.’ What we
    know about each other is also rather fragmented.” (Hans Peter Duerr,
    \textit{Obszönitat und Gewalt}, 1993).

    27. Concerning the paranoid character of fascist behavior, see
    \textit{Anti-Oedipus} by Gilles Deleuze and Félix Guattari.

    28. Gernot Böhme: “The sovereign human being distinguishes itself from the
    autonomous human being not by the increase of dominance over oneself or over
    others. Sovereignty rather means not feeling the need to rule over
    everything. I use the word in the way in which we call someone ‘sovereign’
    who acknowledges the achievements of others, or who accepts defeat.” And:
    “Sovereignty consists especially in accepting the other. … Sovereign human
    beings know that they are not the whole, but a part of the whole, and they
    know how to relate to it.” (\textit{Anthropologie in pragmatischer
    Hinsicht})

    29. Ice-T, \textit{The Ice Opinion} (1994).

    30. Henry Rollins, \textit{One from None} (1987).

    31. Beastie Boys, “The Update”, on: \textit{Ill Communication} (1994).

    32. Rollins, \textit{One from None}.

    33. Böhme, \textit{Anthropologie in pragmatischer Hinsicht}.

    34. “‘Do you know at all what ‘I’ \textit{means} when you say, ‘I know that
    x’ or ‘I go there now’?’ The response to this question can very well be a
    counter-question: ‘Do you ask a record whether it’s only a record, or a
    parrot whether it’s only a parrot?’” (Duerr, \textit{Ni dieu – ni mètre})

    35. John Stuart Mill, \textit{On Liberty} (1859).

    36. Max Stirner, \textit{The Ego and Its Own} (1844).

    37. Gilles Deleuze, \textit{Pourparlers} (1990). It is a misunderstanding to
    see Foucault’s later work in contradiction to his earlier texts. There is no
    “return to the subject.” When Foucault criticized “the subject,” he opposed
    the modern construction of the individual as an autonomous, rational being.
    He never criticized all forms of individuality, or all forms of “exploring
    oneself.” The \textit{History of Sexuality} and other texts from that period
    are simply a continuation of Foucault’s earlier studies. Foucault himself
    said in an interview shortly before his death in 1984: “I don’t think that
    there is a big difference between these books and the former.” Or, as his
    friend and collaborator, Gilles Deleuze, put it: “It is dumb to see a return
    to the subject in this.” (\textit{Pourparlers})

    38. Some “poststructuralist” thinkers found it difficult to free their
    theories from all (explicit or implicit) affinities to contemporary forms of
    domination within capitalist consumer culture. Lyotard’s elaborations on the
    “energetic devil capitalism,” or Deleuze’s and Guattari’s admiration of
    “capitalist schizo streams” appear increasingly problematic, and so do all
    theories about “nomadic subjects,” “permanent changes of identities,” or
    “individual multiplications.” That’s not because the ideas themselves are
    dumb, but because current social realities give them a meaning that takes
    away their revolutionary potential. Based on the thoughtless academic
    mélange of the terms \textit{poststructuralism} and \textit{postmodernity}
    (the latter has become so overexploited by now that a positive reference to
    it appears impossible; Umberto Eco was probably right when he called it a
    “passepartout term,” as it subsumes everything from truly critical
    poststructuralist theory to the lamest liberal nonsense — then again, we
    might still want to use it as a purely descriptive term for a social state
    of arbitrariness), we could say, somewhat paradoxically, that in many ways
    postmodern society has gone beyond poststructuralist theory (Jean
    Baudrillard might have been one of the first to realize this, even if his
    harsh criticism might appear arrogant, but … whatever). Postmodern society
    is characterized by secularization, mobility, pluralism, perpetual change,
    and multiple identities feeding the capitalist machine, which degrades all
    aspects of life to arbitrary consumer goods. This entails no plural
    deconstruction of the individual (which is an explicit poststructuralist
    goal) but conserves the individual as a requirement for the functioning of
    the economic apparatus. The arbitrary individual identities it creates allow
    the individuals to be more flexible, but only in superficial, bourgeois
    ways, beneficial to consumption and production. In that sense, postmodern
    culture truly becomes, in the words of Gilles Lipovetsky, a “vector of
    expansion for individualism” (\textit{L’ère du vide}, 1983). So, as the
    poststructuralist attacks against modern individualism are absorbed by the
    individual’s mantle, poststructuralism loses its sting. We all know the
    result: a life, in which one can “do” (read: consume) anything, and in which
    ethical commitments count for nothing. This casts a dark shadow over the
    writings of many poststructuralist thinkers. It might not be their fault,
    but no theory can escape its social context. The problem is amplified by the
    fact that self-declared postmodern prodigies of poststructuralist thought
    occupy the public arena and offer a poststructuralism/postmodernism that has
    sometimes degenerated into intellectual apologism for parliamentarian
    democracy and the free market. What remains is a purely affirmative theory
    of individuality with no revolutionary potential — often enough, it no
    longer even has such intentions. I agree with the following analysis: “The
    postmodern subject, who is presumably left of subject-hood, seems to be
    mainly the personality constructed by and for technological capital,
    described by the Marxist literary theorist Terry Eagleton as a ‘dispersed,
    decentered network of libidinal attachments, emptied of ethical substance
    and psychical interiority, the ephemeral function of this or that art of
    consumption, media experience, sexual relationship, trend or fashion” (John
    Zerzan, \textit{Future Primitive}, 1994). Seen from that angle,
    poststructuralist/post­modern theory can’t help us at all. I appreciate the
    poststructuralist critique of the subject. But radical theory needs to keep
    a clear distance to the logic of capital, and it must not turn into a
    justification for consumer culture without ethical and political principles.
    It must also go hand in hand with antiindividualistic subjectivities, not
    willing to compromise their principles.

    39. A dynamism of this sort does not make individual values and principles
    relative. For antifascists, antifascist principles remain the same,
    regardless of how their individuality is defined and lived at any given
    time. The ways of how we express our values and principles in everyday life
    might change, but not the values and principles themselves. And just as this
    kind of dynamism doesn’t make our values and principles relative, it doesn’t
    make their expressions in everyday life relative either. Political actions
    that are right today can be wrong tomorrow. But if they were right at the
    time they occurred, no future developments can make them wrong. Changing the
    ways in which we express our values and principles can only make them
    relative if we understand change as an indication of error. But there’s no
    reason for this, unless we want to perpetuate the basis of theoretical and
    practical terror created by Platonic, occidental philosophy. At the end of
    the day, change is all we have, as the subversive intellectual forces of
    ancient Greece already confirmed: Heracliteans, Sophists, and Pyrrhonists.
    We can summarize this in the trivial observation that, in order to have a
    good life, it is right to heat in the winter but wrong to heat in the
    summer. That doesn’t make either wrong per se, it just means that people
    live according to the conditions they find themselves in. I elaborate on
    this because, in the name of postmodernity, we often find justifications of
    individual passivity like the following: without a subject there can’t be
    any principles; no actions are the right ones; those who employ principles
    or make demands are fascists; antifascism, therefore, consists of having no
    principles and taking no action. In reality, this is the end of
    revolutionary politics; it is the “repressive tolerance” Herbert Marcuse
    warned us about: “I call this non-partisan tolerance ‘abstract’ or ‘pure’
    inasmuch as it refrains from taking sides — but in doing so it actually
    protects the already established machinery of discrimination.”
    (\textit{Repressive Tolerance}, 1965) There are no revolutionary politics
    without a commitment to strong values and principles. Not least today, with
    people giving in to an irresponsible ecstasy of consumerism, it would prove
    fatal to give this up. Any antiindividualistic subjectivities we create have
    to be dynamic in the sense described above, while remaining strong and
    uncompromising at the same time. Dynamism must not be confused with
    arbitrariness, otherwise we’ll rob ourselves of all political possibilities.
    I want to underline, however, that this doesn’t mean that I second critiques
    of poststructuralism suggesting that it destroys all possibilities of
    political intervention by attacking the subject. The subject in capitalism
    can indeed be seen as the prison of the individual. It stands against
    transformatory action, since it allows the state to reduce the individual to
    its, well, subject. We require \textit{active subjectivities} with strong
    principles, \textit{not subjects}. There’s a difference. There is a soft
    variety of the subject’s political defense against poststructuralist theory,
    pointing at the pragmatic relevance of being recognized as subjects by the
    law within the modern state system. This is a question that has, for
    example, been raised by feminists who first want to attain full subjecthood
    as women before focusing on the subject’s deconstruction. This can’t be
    disregarded. In fact, a lot of struggles for the recognition as a “full
    subject” are of great importance. But they will remain purely reformist if
    not linked to a radical agenda. For some, this might not be a problem. If,
    for them reforms are enough, they are enough. But for those with different
    ambitions, who don’t want to have any of bourgeois-capitalist society and
    the modern individual, it won’t be. This means that they will have to attack
    the modern subject on an ideological level — which, I believe, can easily go
    together with struggles for attaining subject status on a pragmatic basis.
    The stronger version of the subject’s political defense against
    poststructuralism, however, argues that this is not possible, and indeed
    paradoxical, because we can only act as subjects, and would thus condemn
    ourselves to non-action by attacking the subject. Look, for example, at
    people calling it a “contradiction” that Gayatri Spivak would translate
    Derrida while being a women’s right activist in India. This makes little
    sense. Whoever makes such arguments seems to have missed the point. They
    confuse concrete individuals with their abstract identifications, do not
    differentiate between the subject and subjectivities, and draw the
    ridiculous picture of a poststructuralism supposedly claiming that there are
    no single (individual) human beings. I don’t want my critique of
    poststructuralism to be confused with this. I don’t want to suggest that the
    dissolution of the subject means that we \textit{can’t} act politically
    anymore; but its superficial pseudo-dissolution becomes a justification for
    \textit{not having} to act politically anymore.

    40. Deleuze, \textit{Foucault}.

    41. This is why we always have to consider the context in which supposedly
    individualistic battle cries appear. Even a line like “Do you trust what I
    trust? Me, myself, and I” can simply be a non-conformist commitment. (From
    the Metallica song “Eye of the Beholder” — on \textit{…And Justice for All},
    1988.)

    42. See, for example, Toril Moi’s \textit{Sexual/Textual Politics} (1985).

    43. The enemy confirms the subversive potential of \textit{this} kind of
    individualism: “This sort of ‘individualism’ not only has nothing to do with
    true individualism but may indeed prove a grave obstacle to the smooth
    working of an individualist system. It must remain an open question whether
    a free or individualistic society can be worked successfully if people are
    too ‘individualistic’ in the false sense, if they are too unwilling
    voluntarily to conform to traditions and conventions.” (Richard von Hayek,
    “Individualism: True and False”, 1945)

    44. Foucault, \textit{The Subject and Power}.

    45. ibid.

    46. Peter Kropotkin, \textit{Mutual Aid} (1902).

    47. Two books that write a history of resistance in the 20th-century along
    these lines are Stewart Home’s \textit{The Assault on Culture} (1988), and
    Greil Marcus’s \textit{Lipstick Traces} (1989).

    48. Bakunin, \textit{God and the State}.

    49. Walter Benjamin, \textit{Surrealism} (1929). Despite anarchist
    sympathies, one must not be biased and unfair. Marx understood freedom in a
    similar way, or, at least, criticized “bourgeois freedom” from the same
    perspective: “Only in the community with others, each individual has the
    means to develop its talents; only in community, personal freedom becomes a
    possibility. … In true community, the individual reaches freedom in and by
    association.” (Karl Marx/Friedrich Engels, \textit{The German Ideology},
    1846). In the 1843 essay \textit{On the Jewish Question}, Marx developed a
    well grounded critique of human rights that has not become any less relevant
    today: “Above all, we note the fact that the so-called rights of man, the
    \textit{droits de l’homme} as distinct from the \textit{droits du citoyen},
    are nothing but the rights of a \textit{member of civil society} — i.e., the
    rights of egoistic man, of man separated from other men and from the
    community. … None of the so-called rights of man, therefore, go beyond
    egoistic man, beyond man as a member of civil society — that is, an
    individual withdrawn into himself, into the confines of his private
    interests and private caprice, and separated from the community. In the
    rights of man, he is far from being conceived as a species-being; on the
    contrary, species-life itself, society, appears as a framework external to
    the individuals, as a restriction of their original independence. The sole
    bond holding them together is natural necessity, need and private interest,
    the preservation of their property and their egoistic selves.”

    50. Félix Guattari, \textit{The Three Ecologies}, 1989.
}

% empty page
\newpage
\thispagestyle{empty}
\mbox{}

% author page
\newpage
\thispagestyle{empty}

\vspace*{2.5cm}
\section*{About the author}
\vspace*{4cm}

Gabriel Kuhn is an Austrian-born writer and translator living in Sweden. He
blogs at lefttwothree.org.
